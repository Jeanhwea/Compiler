\section{详细设计}
\subsection{程序结构}
程序的文件列表如下:
\begin{verbatim}
analyse.c analyse.h code.c code.h error.c error.h
global.h main.c nspace.c nspace.h parse.c parse.h
quad.c quad.h scan.c scan.h symtab.c symtab.h
util.c util.h elf.h elf.c
\end{verbatim}
\verb|main| 是主函数入口; \verb|scan| 做词法分析; \verb|parse| 做语法分析生成
语法树; \verb|analyse| 对语法树打印,并进行一些分析;\verb|code| 语义分析,
解析语法树,
生成四元式;\verb|symtab| 管理符号表;\verb|nspace| 管理命名空间;\verb|quad|
发射四元式;\verb|elf| 将四元式生成x86汇编;\verb|util| 是一些工具函数。
\subsection{函数功能}
捡几个比较重要的说。\\
getToken():词法分析,获取单词\\
parse():语法分析,生成语法树\\
analyse():分析语法树\\
code():生成四元式
\subsection{调用关系}
main $->$ parse $->$ analyse $->$ code \\
parse 使用递归下降分析法建立语法树。
里面有很多建立节点的函数,这些节点定义根据文法而来,具体如下:
\begin{verbatim}
static PgmSP PgmB(void);
static BlockSP BlockB(void);
static ConstDecSP ConstDecB(void);
static ConstDefSP ConstDefB(void);
static VarDecSP VarDecB(void);
static VarDefSP VarDefB(void);
static PFDecListSP PFDecListB(void);
static ProcDecSP ProcDecB(void);
static ProcDefSP ProcDefB(void);
static ProcHeadSP ProcHeadB(void);
static FunDecSP FunDecB(void);
static FunDefSP FunDefB(void);
static FunHeadSP FunHeadB(void);
static StmtSP StmtB(void);
static AssignStmtSP AssignStmtB(void);
static IfStmtSP IfStmtB(void);
static RepeStmtSP RepeStmtB(void);
static ForStmtSP ForStmtB(void);
static PcallStmtSP PcallStmtB(void);
static FcallStmtSP FcallStmtB(void);
static CompStmtSP CompStmtB(void);
static ReadStmtSP ReadStmtB(void);
static WriteStmtSP WriteStmtB(void);
static ExprSP ExprB(void);
static TermSP TermB(void);
static FactorSP FactorB(void);
static CondSP CondB(void);
static IdentSP IdentB(IDREADMODE mode);
static ParaListSP ParaListB(void);
static ParaDefSP ParaDefB(void);
static ArgListSP ArgListB(void);
\end{verbatim}
每个节点的数据结构,如下:
\begin{verbatim}
typedef struct _PgmS *PgmSP;
typedef struct _BlockS *BlockSP;
typedef struct _ConstDecS *ConstDecSP;
typedef struct _ConstDefS *ConstDefSP;
typedef struct _VarDecS *VarDecSP;
typedef struct _VarDefS *VarDefSP;
typedef struct _PFDecListS *PFDecListSP;
typedef struct _ProcDecS *ProcDecSP;
typedef struct _ProcDefS *ProcDefSP;
typedef struct _ProcHeadS *ProcHeadSP;
typedef struct _FunDecS *FunDecSP;
typedef struct _FunDefS *FunDefSP;
typedef struct _FunHeadS *FunHeadSP;
typedef struct _StmtS *StmtSP;
typedef struct _AssignStmtS *AssignStmtSP;
typedef struct _IfStmtS *IfStmtSP;
typedef struct _RepeStmtS *RepeStmtSP;
typedef struct _ForStmtS *ForStmtSP;
typedef struct _PcallStmtS *PcallStmtSP;
typedef struct _FcallStmtS *FcallStmtSP;
typedef struct _CompStmtS *CompStmtSP;
typedef struct _ReadStmtS *ReadStmtSP;
typedef struct _WriteStmtS *WriteStmtSP;
typedef struct _ExprS *ExprSP;
typedef struct _TermS *TermSP;
typedef struct _FactorS *FactorSP;
typedef struct _CondS *CondSP;
typedef struct _IdentS *IdentSP;
typedef struct _ParaListS *ParaListSP;
typedef struct _ParaDefS *ParaDefSP;
typedef struct _ArgListS *ArgListSP;
/* declaretion of a bundle of node type */
typedef enum { 
        Fun_PFDec_t , Proc_PFDec_t 
} PFDec_t;
typedef enum {
        Nop_Addop_t, Add_Addop_t, Neg_Addop_t,
        Minus_Addop_t 
} Addop_t;
typedef enum { 
        Nop_Multop_t, Mult_Multop_t, Div_Multop_t 
} Multop_t;
typedef enum { 
        Equ_Rela_t, Neq_Rela_t, Gtt_Rela_t, 
        Geq_Rela_t, Lst_Rela_t, Leq_Rela_t 
} Rela_t;
typedef enum { 
        /* normal identifier type */
        Init_Ident_t, Proc_Ident_t, Int_Fun_Ident_t, 
        Char_Fun_Ident_t,
        /* const identifier type */
        Int_Const_Ident_t, Char_Const_Ident_t,
        /* variable identifier type */
        Int_Var_Ident_t, Char_Var_Ident_t,
        IntArr_Var_Ident_t, CharArr_Var_Ident_t,
        /* parameter identifier type */
        /* call by value */
        Int_Para_Val_Ident_t, Char_Para_Val_Ident_t,
        /* call by address */
        Int_Para_Ref_Ident_t, Char_Para_Ref_Ident_t
} Ident_t;
typedef enum { 
        Assgin_Stmt_t, IF_Stmt_t, Repeat_Stmt_t,
        Pcall_Stmt_t, Comp_Stmt_t, Read_Stmt_t,
        Write_Stmt_t,  For_Stmt_t, 
        Null_Stmt_t 
} Stmt_t;
typedef enum { 
        Norm_Assgin_t, Fun_Assgin_t, Array_Assgin_t 
} Assgin_t;
typedef enum { 
        To_For_t, Downto_For_t 
} For_t;
typedef enum { 
        Id_Factor_t, Array_Factor_t, Unsign_Factor_t,
        Expr_Factor_t, Funcall_Factor_t 
} Factor_t;
typedef enum { 
        StrId_Write_t, Str_Write_t, Id_Write_t 
} Write_t;
/* declaretion of a bundle of struct */
/* Program */
typedef struct _PgmS {
        BlockSP bp;
} PgmS;
/* block */
typedef struct _BlockS {
        ConstDecSP cdp;
        VarDecSP vdp;
        PFDecListSP pfdlp;
        CompStmtSP csp;
} BlockS;
typedef struct _ConstDecS {
        ConstDefSP cdp;
        ConstDecSP next;
} ConstDecS;
typedef struct _ConstDefS {
        IdentSP idp;
} ConstDefS;
typedef struct _VarDecS {
        VarDefSP vdp;
        VarDecSP next;
} VarDecS;
typedef struct _VarDefS {
        IdentSP idp;
        VarDefSP next;
} VarDefS;
typedef struct _PFDecListS {
        PFDec_t type;
        ProcDecSP pdp; 
        FunDecSP fdp; 
        PFDecListSP next;
} PFDecListS;
typedef struct _ProcDecS {
        ProcDefSP pdp;
        ProcDecSP next;
} ProcDecS;
typedef struct _ProcDefS {
        ProcHeadSP php;
        BlockSP bp;
} ProcDefS;
typedef struct _ProcHeadS {
        IdentSP idp;
        ParaListSP plp;
} ProcHeadS;
typedef struct _FunDecS {
        FunDefSP fdp;
        FunDecSP next;
} FunDecS;
typedef struct _FunDefS {
        FunHeadSP fhp;
        BlockSP bp;
} FunDefS;
typedef struct _FunHeadS {
        IdentSP idp;
        ParaListSP plp;
} FunHeadS;
/* statement */
typedef struct _StmtS {
        Stmt_t type;
        AssignStmtSP asp;
        IfStmtSP ifp;
        RepeStmtSP rpp;
        ForStmtSP frp;
        PcallStmtSP pcp;
        CompStmtSP cpp;
        ReadStmtSP rdp;
        WriteStmtSP wtp;
} StmtS;
typedef struct _AssignStmtS {
        Assgin_t type;
        IdentSP idp;
        ExprSP lep;
        ExprSP rep;
} AssignStmtS;
typedef struct _IfStmtS {
        CondSP cp;
        /* then */
        StmtSP tp;
        /* else */
        StmtSP ep;
} IfStmtS;
typedef struct _RepeStmtS {
        StmtSP sp;
        CondSP cp;
} RepeStmtS;
typedef struct _ForStmtS {
        For_t type;
        IdentSP idp;
        ExprSP lep;
        ExprSP rep;
        StmtSP sp;
} ForStmtS;
typedef struct _PcallStmtS {
        IdentSP idp;
        ArgListSP alp;
} PcallStmtS;
typedef struct _FcallStmtS {
        IdentSP idp;
        ArgListSP alp;
} FcallStmtS;
typedef struct _CompStmtS {
        StmtSP sp;
        CompStmtSP next;
} CompStmtS;
typedef struct _ReadStmtS {
        IdentSP idp;
        ReadStmtSP next;
} ReadStmtS;
typedef struct _WriteStmtS {
        Write_t type;
        /* string pointer */
        char *sp;
        ExprSP ep;
} WriteStmtS;
/* expression term factor condition */
typedef struct _ExprS {
        Addop_t op;
        TermSP tp;
        ExprSP next;
} ExprS;
typedef struct _TermS {
        Multop_t op;
        FactorSP fp;
        TermSP next;
} TermS;
typedef struct _FactorS {
        Factor_t type;
        IdentSP idp;
        ExprSP ep;
        /* unsign int */
        int usi;
        FcallStmtSP fcsp;
} FactorS;
typedef struct _CondS {
        ExprSP lep;
        Rela_t op;
        ExprSP rep;
} CondS;
/* ident parameter argument*/
typedef struct _IdentS {
        Ident_t type;
        char *name;
        int val;
        int length;
        int line;
} IdentS;
typedef struct _ParaListS {
        ParaDefSP pdp;
        ParaListSP next;
} ParaListS;
typedef struct _ParaDefS {
        IdentSP idp;
        ParaDefSP next;
} ParaDefS;
typedef struct _ArgListS {
        ExprSP ep;
        ArgListSP next;
} ArgListS;
\end{verbatim}
analyse 做语法分析, code 遍历语法树各个节点,生成四元式。数据结构有点
多,就不粘贴了。
\subsection{符号表管理}
符号表采用栈式符号表,每个函数或过程进入是申请符号表,退出时弹出。
每个符号表项使用符号名hash查找,从栈顶往栈底查。
数据结构如下:
\begin{verbatim}
/* hash size */
extern int HASHSIZE;
/* hash shift */
extern int SHIFT;

typedef struct _SymTabS *SymTabSP;
typedef struct _SymLineS *SymLineSP;
typedef struct _SymBucketS *SymBucketSP;
typedef struct _SymTabES *SymTabESP;

typedef enum { 表项类型
        Nop_Obj_t, Const_Obj_t, Var_Obj_t,
        Proc_Obj_t, Fun_Obj_t, Array_Obj_t,
        Para_Obj_t, Tmp_Obj_t
} Obj_t;

typedef enum { 符号类型
        Int_Type_t, Char_Type_t, Nop_Type_t
} Type_t;

/**
 * symbol table stack 
 * is a stack that
 * manage symbol table in
 * each function
 */
typedef struct _SymTabS { 符号表链,使用栈式符号表
        SymBucketSP *sbp;       hash 符号索引表头
        char *ns;               // namespace for a block
        SymTabSP prev;
        SymTabSP next;
} SymTabS;

/**
 * store which line a varible
 * be referenced
 */
typedef struct _SymLineS { 符号的行号记录
        int lineno;
        SymLineSP next;
} SymLineS;

/**
 * symbol table entry 
 * bucket list
 */
typedef struct _SymBucketS {
        SymTabESP ep;           // element infomation
        SymBucketSP next;
} SymBucketS;

typedef struct _SymTabES { 符号表项
        char *name;             // identifier name
        char *label;            // namespace label
        int val;                // array length
                                // or const value
        SymLineSP lines;        // referenced lines
        Obj_t obj;              // object type
        Type_t type;            // type 
        SymTabSP stp;           // point to symbol table
} SymTabES;
对应的符号表操作
SymTabSP pop(void);
SymTabSP newstab(void);
void push(SymTabSP);
char *mkUsi(int);
SymTabESP sym_insert_const(IdentSP);
SymTabESP sym_insert_var(IdentSP);
SymTabESP sym_insert_para(IdentSP);
SymTabESP sym_insert_fun(IdentSP, ParaListSP);
SymTabESP sym_insert_proc(IdentSP, ParaListSP);
SymTabESP sym_insert_tmp();
char *genLabel(void);
SymTabESP sym_lookup(char *);
void printTab(SymTabSP);
\end{verbatim}
\section{存储分配}
暂定常量,数组和变量静态分配
临时变量和参数压每个函数栈
\clearpage
\section{四元式设计}
四元式设计如下:\\
算术指令:\\
\begin{tabular}{r|c}
	\hline
	ADD  r, s, d & d = r + s \\ \hline
	SUB  r, s, d & d = r - s \\ \hline
	MUL  r, s, d & d = r * s \\ \hline
	DIV  r, s, d & d = r / s \\ \hline
\end{tabular} \\
关系指令:\\
\begin{tabular}{r|c}
	\hline
	EQU  r, s, d & d = (r == s)?1:0 \\ \hline
	NEQ  r, s, d & d = (r != s)?1:0 \\ \hline
	GTT  r, s, d & d = (r $>$ s)?1:0 \\ \hline
	GEQ  r, s, d & d = (r $>=$ s)?1:0 \\ \hline
	LST  r, s, d & d = (r $<$ s)?1:0 \\ \hline
	LEQ  r, s, d & d = (r $<=$ s)?1:0 \\ \hline
\end{tabular} \\
比较指令:\\
\begin{tabular}{r|c}
	\hline
	BRZ  r, , label & jump to label if r = 0 \\ \hline
	BNZ  r, , label & jump to label if r != 0 \\ \hline
	BGT  r, s, label & jump to label if r $>$ s \\ \hline
	BLT  r, s, label & jump to label if r $<$ s \\ \hline
\end{tabular}\\
跳转指令:\\
\begin{tabular}{r|c}
	\hline
	JMP  , , label & jump to label\\ \hline
\end{tabular}\\
栈操作指令:\\
\begin{tabular}{r|c}
	\hline
	PUSH  , , d & push d into stack\\ \hline
	POP  , , d & pop d out of stack\\ \hline
\end{tabular}\\
函数调用:\\
\begin{tabular}{r|c}
	\hline
	CALL  fun, , d & d = fun()\\ \hline
	RET  , , & set function return value\\ \hline
	END  , , & end of a function\\ \hline
\end{tabular}\\
IO指令:\\
\begin{tabular}{r|c}
	\hline
	READ  , , d & d = fun()\\ \hline
	WRITE  r, , & write a identifier\\ \hline
	WRITES  , s, & write a string\\ \hline
\end{tabular}
\section{错误处理}
\begin{enumerate}
        \item 该标识符未定义
        \item 识符重复定义
        \item 应是标识符
        \item 应是‘ ( ’
        \item 应是‘ : ’,在说明类型时必须有此冒号
        \item 非法符号,编译将跳读该符号和后面的某些符号
        \item 形式参数表中,形参说明应该以标识符或var开头
        \item 应是of
        \item 应是‘ ( ’
        \item 类型定义必须以array或基本类型开头
        \item 应是‘ [ ’
        \item 应是‘ ] ’
        \item 应是‘ ; ’
        \item 函数结果必须是integer或char类型
        \item 应是‘ = ’,‘ := ’只能在赋值语句里使用,而不能在说明中使用
        \item 在if后面必须是条件
        \item 在for后面的循环变量只能是integer或char型
        \item for语句中初值或终值表达式必须与循环变量类型相同
        \item 数太大
        \item 程序结尾是‘ . ’,请检查相应的begin和end
        \item 非法字符
        \item 在常量定义中,等号后面必须是常数
        \item 下标表达式类型必须是integer型
        \item 数组说明中,下界小于0,类型不是integer型。
        \item 没有这样的数组
        \item 该算术表达式的类型不合法。注意,数组整体不能作为算术运算的操作数
        \item 实参和对应形参类型应相同
        \item 应是变量
        \item 实参个数与形参个数不等
        \item 表达式中不能出现类型或过程标识符
        \item 应是变量或过程/函数标识符
        \item 在赋值语句中被赋值变量应与表达式类型相同
        \item 数组越界
        \item 应是常量
        \item 应是‘ := ’
        \item 应是then
        \item 应是do
        \item 应是to
        \item 应是begin
        \item 应是end
        \item 因子必须以标识符、常量或‘ ( ’开始
\end{enumerate}
\section{测试程序}
做语法分析的测试程序:
\begin{verbatim}
{ syntax test }
const 
        a = 1, 
        b = -3, 
        c = +28, 
        let = 'c', 
        nuh = '4';
var 
        v : integer;
        head, ch2:char; 
        q, p:integer; 
        Aint:array[3] of integer;
        Achar:array[5] of char;
{ pass by value}
function FunctionName1(funvar1, funvar2:char): integer;
begin
        A[0] := (funvar1 * funvar2 / 6 ) + 4;
        FunctionName1 := a * ( 4 + funvar2)
end;
procedure ProcedureName1();
        { pass by reference }
        procedure ProcedureName2(var num:integer);
        const silly = -2, foo = 'p';
        begin end;
begin end;
function FunctionName2(var x, y: integer):char;
        var z:integer;
        function FunctionName3(): integer;
        begin { null statement } end;
        begin
                { array assign }
                A[2] := - x + y;
                FunctionName2 := z - y 
        end;
{ main }
begin
        { read test }
        read(p);
        read(p, q);
        ProcedureName1();
        { if test }
        if head > 0 then
                write("yes")
        else
                write(head);
        { normal assign }
        p := a * 9 / r;
        r := r - 4;
        { for test }
        for i := 20 downto + 10 * a do
                sum := sum + i + pp;
        FunctionName2(pa,ra);
        for i := 1 to 10 do
                sum := sum * i;
        { repeate test }
        repeat
                if head <= 4 then
                        write (" head = ",head)
        until head <> 7
end.
\end{verbatim}
冒泡排序,进行语义分析
\begin{verbatim}
{program bubble sort;}
var
        a:array[5] of integer {= (4, 5, 2, 7, 0)};
        i, j: integer;
procedure swap(var x,y:integer);
        var
                t: integer;
        begin
                t := x;
                x := y;
                y := t
        end;
begin { main }
        for i := 1 to 5 do
        begin
                for j := i to 5 do
                begin
                        if a[i] > a[j] then
                        begin
                                swap(a[i], a[j])
                        end;
                end;
        end;
        for i := 1 to 5 do
        begin
                write(a[i])
                {write(' ')}
        end
        {writeln()}
end.
\end{verbatim}
基本语句的测试
\begin{verbatim}
const INT = 9;
var i, a, A, B, C, D, E: integer;
        AAA:array[100] of integer;
function FOO(): integer;
var A, B:integer;
begin
        a := -32+FOO()-AAA[A+B]*INT-(C-E);
end;
begin
        if E < 0 then
        begin
                a := 88
        end;
        repeat
                a := -32+FOO()-AAA[A+B]*INT-(C-E)
        until a>A;
        for i := 1 to 3 do
        begin
                i := i + 1
        end;
end.
\end{verbatim}
