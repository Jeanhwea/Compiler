\section{详细设计}
\subsection{程序结构}
程序的文件列表如下:
\begin{verbatim}
analyse.c analyse.h code.c code.h error.c error.h
global.h main.c nspace.c nspace.h parse.c parse.h
quad.c quad.h scan.c scan.h symtab.c symtab.h
util.c util.h elf.h elf.c
\end{verbatim}
\verb|main| 是主函数入口; \verb|scan| 做词法分析; \verb|parse| 做语法分析生成
语法树; \verb|analyse| 对语法树打印,并进行一些分析;\verb|code| 语义分析,
解析语法树,
生成四元式;\verb|symtab| 管理符号表;\verb|nspace| 管理命名空间;\verb|quad|
发射四元式;\verb|elf| 将四元式生成x86汇编;\verb|util| 是一些工具函数。
\subsection{函数功能}
捡几个比较重要的说。\\
getToken():词法分析,获取单词\\
parse():语法分析,生成语法树\\
analyse():分析语法树\\
code():生成四元式\\
elf():生成elf文本格式的x86汇编
\subsection{调用关系}
main $->$ parse $->$ analyse $->$ code \\
parse 使用递归下降分析法建立语法树。
里面有很多建立节点的函数,这些节点定义根据文法而来,具体如下:
\begin{verbatim}
static PgmSP PgmB(void);
static BlockSP BlockB(void);
static ConstDecSP ConstDecB(void);
static ConstDefSP ConstDefB(void);
static VarDecSP VarDecB(void);
static VarDefSP VarDefB(void);
static PFDecListSP PFDecListB(void);
static ProcDecSP ProcDecB(void);
...
static TermSP TermB(void);
static FactorSP FactorB(void);
static CondSP CondB(void);
static IdentSP IdentB(IDREADMODE mode);
static ParaListSP ParaListB(void);
static ParaDefSP ParaDefB(void);
static ArgListSP ArgListB(void);
\end{verbatim}
每个节点的数据结构,如下:
\begin{verbatim}
typedef struct _PgmS *PgmSP;
typedef struct _BlockS *BlockSP;
typedef struct _ConstDecS *ConstDecSP;
typedef struct _ConstDefS *ConstDefSP;
typedef struct _VarDecS *VarDecSP;
typedef struct _VarDefS *VarDefSP;
typedef struct _PFDecListS *PFDecListSP;
typedef struct _ProcDecS *ProcDecSP;
...
typedef struct _FactorS *FactorSP;
typedef struct _CondS *CondSP;
typedef struct _IdentS *IdentSP;
typedef struct _ParaListS *ParaListSP;
typedef struct _ParaDefS *ParaDefSP;
typedef struct _ArgListS *ArgListSP;
/* declaretion of a bundle of node type */
typedef enum { 
  Fun_PFDec_t , Proc_PFDec_t 
} PFDec_t;
...
typedef enum { 
  /* normal identifier type */
  Init_Ident_t, Proc_Ident_t, Int_Fun_Ident_t, 
  Char_Fun_Ident_t,
  /* const identifier type */
  Int_Const_Ident_t, Char_Const_Ident_t,
  /* variable identifier type */
  Int_Var_Ident_t, Char_Var_Ident_t,
  IntArr_Var_Ident_t, CharArr_Var_Ident_t,
  /* parameter identifier type */
  /* call by value */
  Int_Para_Val_Ident_t, Char_Para_Val_Ident_t,
  /* call by address */
  Int_Para_Ref_Ident_t, Char_Para_Ref_Ident_t
} Ident_t;
...
typedef enum { 
  StrId_Write_t, Str_Write_t, Id_Write_t 
} Write_t;
/* declaretion of a bundle of struct */
/* Program */
typedef struct _PgmS {
  BlockSP bp;
} PgmS;
/* block */
typedef struct _BlockS {
  ConstDecSP cdp;
  VarDecSP vdp;
  PFDecListSP pfdlp;
  CompStmtSP csp;
} BlockS;
typedef struct _ConstDecS {
  ConstDefSP cdp;
  ConstDecSP next;
} ConstDecS;
typedef struct _ConstDefS {
  IdentSP idp;
} ConstDefS;
...
typedef struct _ArgListS {
  ExprSP ep;
  ArgListSP next;
} ArgListS;
\end{verbatim}
analyse 做语法分析, code 遍历语法树各个节点,生成四元式。数据结构有点
多,就不粘贴了。
\subsection{符号表管理}
符号表采用栈式符号表,每个函数或过程进入是申请符号表,退出时弹出。
每个符号表项使用符号名hash查找,从栈顶往栈底查。
数据结构如下:
\begin{verbatim}
/* hash size */
extern int HASHSIZE;
/* hash shift */
extern int SHIFT;

typedef struct _SymTabS *SymTabSP;
typedef struct _SymLineS *SymLineSP;
typedef struct _SymBucketS *SymBucketSP;
typedef struct _SymTabES *SymTabESP;

typedef enum { 表项类型
  Nop_Obj_t, Const_Obj_t, Var_Obj_t,
  Proc_Obj_t, Fun_Obj_t, Array_Obj_t,
  Para_Obj_t, Tmp_Obj_t
} Obj_t;

typedef enum { 符号类型
  Int_Type_t, Char_Type_t, Nop_Type_t
} Type_t;

/**
 * symbol table stack 
 * is a stack that
 * manage symbol table in
 * each function
 */
typedef struct _SymTabS { 符号表链,使用栈式符号表
  SymBucketSP *sbp;       hash 符号索引表头
  char *ns;               // namespace for a block
  SymTabSP prev;
  SymTabSP next;
} SymTabS;

/**
 * store which line a varible
 * be referenced
 */
typedef struct _SymLineS { 符号的行号记录
  int lineno;
  SymLineSP next;
} SymLineS;

/**
 * symbol table entry 
 * bucket list
 */
typedef struct _SymBucketS {
  SymTabESP ep;           // element infomation
  SymBucketSP next;
} SymBucketS;

typedef struct _SymTabES { 符号表项
  char *name;             // identifier name
  char *label;            // namespace label
  int val;                // array length
        // or const value
  SymLineSP lines;        // referenced lines
  Obj_t obj;              // object type
  Type_t type;            // type 
  SymTabSP stp;           // point to symbol table
} SymTabES;
对应的符号表操作
SymTabSP pop(void);
SymTabSP newstab(void);
void push(SymTabSP);
char *mkUsi(int);
SymTabESP sym_insert_const(IdentSP);
SymTabESP sym_insert_var(IdentSP);
SymTabESP sym_insert_para(IdentSP);
SymTabESP sym_insert_fun(IdentSP, ParaListSP);
SymTabESP sym_insert_proc(IdentSP, ParaListSP);
SymTabESP sym_insert_tmp();
char *genLabel(void);
SymTabESP sym_lookup(char *);
void printTab(SymTabSP);
\end{verbatim}
\subsection{存储分配}
常量编译时候直接嵌入汇编里,无需分配空间;变量动态分配存储空间
见下面:运行栈示意图的 variables 部分,数组也能分配在这个区域,
不同的是数组是分配一段连续的内存空间。临时变量动态分配在变量空间
下面。参数压每个函数栈。那些 saved ebp 是 display 区。
\begin{verbatim}
/**
 *             RUNTIME STACK OVERVIEW
 *
 *                   high addr
 *               |    ... ...   |
 *               |    ... ...   |
 *               | parameter(1) |
 *               | parameter(2) |       ||
 *               | parameter(3) |       ||
 *               | saved ebp(3) |       ||
 *               | saved ebp(2) |       ||
 *      ebp+8 -> | saved ebp(1) |       ||
 *      ebp+4 -> | return addr  |       || stack
 *        ebp -> | prev ebp     |       || pointer
 *               | return value |       || increase
 *               | variables(1) |       ||
 *               | variables(2) |      \||/
 *               | variables(3) |       \/
 *               | temporary(1) |
 *               | temporary(2) |
 *               | temporary(3) |
 *               | saved ebx    |
 *               | saved esi    |
 *               | saved edi    |
 *               |    ... ...   |
 *        esp -> |    ... ...   |
 *
 **                   low addr
 */
\end{verbatim}
\subsection{四元式设计}
四元式设计如下:\\
\vbox{}\\
算术指令:\\
\begin{tabular}{r|c}
  \hline
  ADD  r, s, d & d = r + s \\ \hline
  SUB  r, s, d & d = r - s \\ \hline
  MUL  r, s, d & d = r * s \\ \hline
  DIV  r, s, d & d = r / s \\ \hline
  INC  -, -, d & d++       \\ \hline
  DEC  -, -, d & d--       \\ \hline
  NEG  r, -, d & d = -r    \\ \hline
\end{tabular} \\
\vbox{}\\
存取赋值指令:\\
\begin{tabular}{r|c}
  \hline
  LOAD  r, -, d & load the value of r into d \\ \hline
  ASS   r, -, d & assign r into d; d:=r \\ \hline
  ASSA  r, s, d & d := r[s] \\ \hline
\end{tabular} \\
\vbox{}\\
条件指令:\\
\begin{tabular}{r|c}
  \hline
  EQU  r, s, d & brance to label d if (r == s) \\ \hline
  NEQ  r, s, d & brance to label d if (r != s) \\ \hline
  GTT  r, s, d & brance to label d if (r $>$ s)\\ \hline
  GEQ  r, s, d & brance to label d if (r $>=$ s)\\ \hline
  LST  r, s, d & brance to label d if (r $<$ s) \\ \hline
  LEQ  r, s, d & brance to label d if (r $<=$ s)\\ \hline
\end{tabular} \\
\vbox{}\\
无条件跳转指令:\\
\begin{tabular}{r|c}
  \hline
  JMP  -, -, d(label) & jump to label\\ \hline
\end{tabular}\\
\vbox{}\\
栈操作指令:\\
\begin{tabular}{r|c}
  \hline
  PUSH  -, -, d & push d into stack\\ \hline
  PUSHA -, -, d & push the address of d into stack \\ \hline
  POP   -, -, d & pop d out of stack\\ \hline
\end{tabular}\\
\vbox{}\\
函数调用:\\
\begin{tabular}{r|c}
  \hline
  LABEL -, -, d(label) & label for brance \\ \hline
  CALL  r(funlabel), , d & d = r()\\ \hline
  SRET  -, -, d & set function return value\\ \hline
  ENTER -, -, d(funlabel) & enterance of a function \\ \hline
  FIN   -, -, -& finish a function\\ \hline
\end{tabular}\\
\vbox{}\\
IO指令:\\
\begin{tabular}{r|c}
  \hline
  READ  -, -, d & read a integer\\ \hline
  READC -, -, d & read a char\\ \hline
  WRI   -, -, d & write a integer \\ \hline
  WRC   -, -, d & write a char \\ \hline
  WRS   -, -, d & write a string\\ \hline
\end{tabular}
\subsection{优化方案}
\subsubsection{DAG图}
基本块结构如下:
\begin{verbatim}
typedef struct _BBS *BBSP;
typedef struct _BBListS *BBListSP;

typedef struct _BBListS {
  BBSP bbp;
  BBListSP next;
} BBListS;

typedef struct _BBS {
  int id;
  // point to a function or procedure quadruples
  QuadSP qhead;
  QuadSP qtail;
  QuadSP scope; 
  QuadSP first; // first position of basic block
  QuadSP last;  // last position of basic block
} BBS;
\end{verbatim}
DAG图的优化见 \verb|dag.c| 文件。主要是先建立dag图
在使用递归式重DAG图推到代码序列。
\subsection{错误处理}
\begin{enumerate}
  \item 该标识符未定义
  \item 识符重复定义
  \item 应是标识符
  \item 应是‘ ( ’
  \item 应是‘ : ’,在说明类型时必须有此冒号
  \item 非法符号,编译将跳读该符号和后面的某些符号
  \item 形式参数表中,形参说明应该以标识符或var开头
  \item 应是of
  \item 应是‘ ( ’
  \item 类型定义必须以array或基本类型开头
  \item 应是‘ [ ’
  \item 应是‘ ] ’
  \item 应是‘ ; ’
  \item 函数结果必须是integer或char类型
  \item 应是‘ = ’,‘ := ’只能在赋值语句里使用,而不能在说明中使用
  \item 在if后面必须是条件
  \item 在for后面的循环变量只能是integer或char型
  \item for语句中初值或终值表达式必须与循环变量类型相同
  \item 数太大
  \item 程序结尾是‘ . ’,请检查相应的begin和end
  \item 非法字符
  \item 在常量定义中,等号后面必须是常数
  \item 下标表达式类型必须是integer型
  \item 数组说明中,下界小于0,类型不是integer型。
  \item 没有这样的数组
  \item 该算术表达式的类型不合法。注意,数组整体不能作为算术运算的操作数
  \item 实参和对应形参类型应相同
  \item 应是变量
  \item 实参个数与形参个数不等
  \item 表达式中不能出现类型或过程标识符
  \item 应是变量或过程/函数标识符
  \item 在赋值语句中被赋值变量应与表达式类型相同
  \item 数组越界
  \item 应是常量
  \item 应是‘ := ’
  \item 应是then
  \item 应是do
  \item 应是to
  \item 应是begin
  \item 应是end
  \item 因子必须以标识符、常量或‘ ( ’开始
\end{enumerate}
